%% start of file `template.tex'.
%% Copyright 2006-2013 Xavier Danaux (xdanaux@gmail.com).
%
% This work may be distributed and/or modified under the
% conditions of the LaTeX Project Public License version 1.3c,
% available at http://www.latex-project.org/lppl/.
%Version for spanish users, by dgarhdez

\documentclass[11pt,a4paper,roman]{moderncv}        % possible options include font size ('10pt', '11pt' and '12pt'), paper size ('a4paper', 'letterpaper', 'a5paper', 'legalpaper', 'executivepaper' and 'landscape') and font family ('sans' and 'roman')
\usepackage[english]{babel}


% moderncv themes
\moderncvstyle{classic}                            % style options are 'casual' (default), 'classic', 'oldstyle' and 'banking'
\moderncvcolor{green}                              % color options 'blue' (default), 'orange', 'green', 'red', 'purple', 'grey' and 'black'
%\renewcommand{\familydefault}{\sfdefault}         % to set the default font; use '\sfdefault' for the default sans serif font, '\rmdefault' for the default roman one, or any tex font name
%\nopagenumbers{}                                  % uncomment to suppress automatic page numbering for CVs longer than one page

% character encoding
\usepackage[utf8]{inputenc}                       % if you are not using xelatex ou lualatex, replace by the encoding you are using
%\usepackage{CJKutf8}                              % if you need to use CJK to typeset your resume in Chinese, Japanese or Korean

% adjust the page margins
\usepackage[scale=0.75]{geometry}
%\setlength{\hintscolumnwidth}{3cm}                % if you want to change the width of the column with the dates
%\setlength{\makecvtitlenamewidth}{10cm}           % for the 'classic' style, if you want to force the width allocated to your name and avoid line breaks. be careful though, the length is normally calculated to avoid any overlap with your personal info; use this at your own typographical risks...

% personal data
\name{Evgenii}{Safronov}
%\title{Resumé title}                               % optional, remove / comment the line if not wanted
%\address{Dirección}{CP, Ciudad}{País}% optional, remove / comment the line if not wanted; the "postcode city" and and "country" arguments can be omitted or provided empty
%\phone[mobile]{000-000-000-000}                   % optional, remove / comment the line if not wanted
%\phone[fixed]{+2~(345)~678~901}                    % optional, remove / comment the line if not wanted
%\phone[fax]{+3~(456)~789~012}                      % optional, remove / comment the line if not wanted
%\email{mailmailmail@gmail.com}                               % optional, remove / comment the line if not wanted
%\homepage{www.johndoe.com}                         % optional, remove / comment the line if not wanted
%\extrainfo{additional information}                 % optional, remove / comment the line if not wanted
%\photo[64pt][0.4pt]{picture}                       % optional, remove / comment the line if not wanted; '64pt' is the height the picture must be resized to, 0.4pt is the thickness of the frame around it (put it to 0pt for no frame) and 'picture' is the name of the picture file
%\quote{Some quote}                                 % optional, remove / comment the line if not wanted

% to show numerical labels in the bibliography (default is to show no labels); only useful if you make citations in your resume
%\makeatletter
%\renewcommand*{\bibliographyitemlabel}{\@biblabel{\arabic{enumiv}}}
%\makeatother
%\renewcommand*{\bibliographyitemlabel}{[\arabic{enumiv}]}% CONSIDER REPLACING THE ABOVE BY THIS

% bibliography with mutiple entries
%\usepackage{multibib}
%\newcites{book,misc}{{Books},{Others}}
%----------------------------------------------------------------------------------
%            content
%----------------------------------------------------------------------------------
\begin{document}
%-----       letter       ---------------------------------------------------------
% recipient data
\recipient{Skoltech}{Space Systems MSc program}
\date{\today}
\opening{Dear Skoltech,}
\closing{Hope I delivered my sincere interest in robotics and study in Skoltech.}
%\enclosure[Adjunto]{CV}          % use an optional argument to use a string other than "Enclosure", or redefine \enclname
\makelettertitle

Few days ago I received bachelor degree with honors in Moscow Institute of Physics and Technology and now I'm applying for Skoltech MSc program. In this letter I'm going to tell you why I decided to study robotics, why I chose Skoltech and what do I hope to gain by completing a degree in Space Center. 
\newline
\newline
I think that I'm lucky guy, because during last term of my study I prepared for Eurobot competition and found out that robotics is what I can work on from dawn to dusk.\newline
\newline
\textbf{Why robotics?} Robotics is a multidisciplinary subject -- it involves mechanical engineering, electrical engineering, programming and especially AI. Don't forget about developing proper human-robot interaction, design and so one. Robots will definitely change the human environment at their homes and public places as they are performing ``4th industrial revolution''. Nowadays electrical components of future robots become more powerful and cheaper due to the wide distribution of smartphones and computers. Sum this up with recent breakthrough in AI -- robotics puzzle is about to be collected! Another important property of robotics is that you \textit{create} robots. I realised that I am the most strongly motivated to work when I create something by my own, not when I do some specific and even interesting task. Richard Feynman -- one of the most famous physicists -- mentioned in his famous autobiography that he enjoyed physics because he used to play with it. I don't pretend to be a one tenth cool as he was but this formula seems to work for me in robotics. 
\newline
\newline
\textbf{Why Skoltech?} Skoltech provides unique opportunity to study and do robotics. While there are respected computer science/electrical engineering/mechanical engineering MSc programs in MIPT and other classical russian universities, robotics requires additional attention, independent lab and specific courses -- as implemented in Skoltech. Space-related courses are also helpful because many tasks in autonomous systems both for space and earth can be solved in the same manner. 
\newline
\newline 
\textbf{What I expect} from next two years in Skoltech? A lot.\newline In my opinion, successful MSc student shouldn't focus only at courses, but pay most of his attention to developing a project. I'm going to work on it since my first days in Skoltech. Hope we will complete some working solution with a commercial potential during next two years. It is the first step that is troublesome, I can't be sure that first project will be successful. However, after completing a MSc degree in SkolTech I hope I will be ready to develop complex mobile robot. This means acquiring wide knowledge of recent results in robotics, expierence of planning, developing and testing real solution. I want to work hard and honestly as it was through my bachelor degree in MIPT. Taking opportunities of Skoltech into account, wasting of time is senseless and irrational! 
\newline
\newline
Bachelor degree in MIPT provided me with required mathematical background -- I'd like to mention linear algebra and probability theory. The main quality I trained is the ability to learn large amount of technics, new information or theory. I joined Eurobot competition to practice in basics of robotics, had to overcome a lot of obstacles and finished a robot which participated in Russian finals. Despite of studing mainly physics in University I always had a programming practice  -- I took additional courses and finally one of my thesis goals was to develop a Python framework for physical simulations. Thus I suppose that I meet all requirements for degree and can work from first days.
%\vspace{0.5cm}


\makeletterclosing

\end{document}


%% end of file `template.tex'.
