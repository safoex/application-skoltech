%% start of file `template.tex'.
%% Copyright 2006-2013 Xavier Danaux (xdanaux@gmail.com).
%
% This work may be distributed and/or modified under the
% conditions of the LaTeX Project Public License version 1.3c,
% available at http://www.latex-project.org/lppl/.
%Version for spanish users, by dgarhdez

\documentclass[11pt,a4paper,roman]{moderncv}        % possible options include font size ('10pt', '11pt' and '12pt'), paper size ('a4paper', 'letterpaper', 'a5paper', 'legalpaper', 'executivepaper' and 'landscape') and font family ('sans' and 'roman')
\usepackage[english]{babel}
\usepackage{xcolor}
\usepackage{ wasysym }
\definecolor{RED}{HTML}{FF0000}

% moderncv themes
\moderncvstyle{classic}                            % style options are 'casual' (default), 'classic', 'oldstyle' and 'banking'
\moderncvcolor{green}                              % color options 'blue' (default), 'orange', 'green', 'red', 'purple', 'grey' and 'black'
%\renewcommand{\familydefault}{\sfdefault}         % to set the default font; use '\sfdefault' for the default sans serif font, '\rmdefault' for the default roman one, or any tex font name
%\nopagenumbers{}                                  % uncomment to suppress automatic page numbering for CVs longer than one page

% character encoding
\usepackage[utf8]{inputenc}                       % if you are not using xelatex ou lualatex, replace by the encoding you are using
%\usepackage{CJKutf8}                              % if you need to use CJK to typeset your resume in Chinese, Japanese or Korean

% adjust the page margins
\usepackage[scale=0.75]{geometry}
%\setlength{\hintscolumnwidth}{3cm}                % if you want to change the width of the column with the dates
%\setlength{\makecvtitlenamewidth}{10cm}           % for the 'classic' style, if you want to force the width allocated to your name and avoid line breaks. be careful though, the length is normally calculated to avoid any overlap with your personal info; use this at your own typographical risks...

% personal data
\name{Evgenii}{Safronov}
%\title{Resumé title}                               % optional, remove / comment the line if not wanted
%\address{Dirección}{CP, Ciudad}{País}% optional, remove / comment the line if not wanted; the "postcode city" and and "country" arguments can be omitted or provided empty
%\phone[mobile]{000-000-000-000}                   % optional, remove / comment the line if not wanted
%\phone[fixed]{+2~(345)~678~901}                    % optional, remove / comment the line if not wanted
%\phone[fax]{+3~(456)~789~012}                      % optional, remove / comment the line if not wanted
%\email{mailmailmail@gmail.com}                               % optional, remove / comment the line if not wanted
%\homepage{www.johndoe.com}                         % optional, remove / comment the line if not wanted
%\extrainfo{additional information}                 % optional, remove / comment the line if not wanted
%\photo[64pt][0.4pt]{picture}                       % optional, remove / comment the line if not wanted; '64pt' is the height the picture must be resized to, 0.4pt is the thickness of the frame around it (put it to 0pt for no frame) and 'picture' is the name of the picture file
%\quote{Some quote}                                 % optional, remove / comment the line if not wanted

% to show numerical labels in the bibliography (default is to show no labels); only useful if you make citations in your resume
%\makeatletter
%\renewcommand*{\bibliographyitemlabel}{\@biblabel{\arabic{enumiv}}}
%\makeatother
%\renewcommand*{\bibliographyitemlabel}{[\arabic{enumiv}]}% CONSIDER REPLACING THE ABOVE BY THIS

% bibliography with mutiple entries
%\usepackage{multibib}
%\newcites{book,misc}{{Books},{Others}}
%----------------------------------------------------------------------------------
%            content
%----------------------------------------------------------------------------------
\begin{document}
%-----       letter       ---------------------------------------------------------
% recipient data
\recipient{Open Robotics}{info@openrobotics.org}
\date{\today}
\opening{Dear Open Robotics,}
\closing{Hope I delivered my sincere interest in your internship.}
%\enclosure[Adjunto]{CV}          % use an optional argument to use a string other than "Enclosure", or redefine \enclname
\makelettertitle
since I use ROS (and program with its libraries) \textit{every day} I couldn't miss opportunity to apply for this amazing intership. Now I'm a captain of Skoltech's Eurobot team (\url{http://www.eurobot.org/}). This contest includes full development of 2 mobile autonomous robots. Our code architecture is built around ROS (see here \url{https://github.com/SkoltechRobotics/ros-eurobot-2018}, in development). I'm really inspired by the work you are doing, it cuts time of development and simplify concurrent engineering inside team.
\newline
\newline
Last two years I have been coding primarily on Python. And now my code for Eurobot is also written on Python. However, I've started to learn C++ in school (7 years ago!) and continued in university. I've got bachelor degree with honors (GPA 4.92/5) from MIPT and always got excellent grades for programming subjects. MIPT is one of the best russian universities in math \& programming. For example, MIPT team was awarded with Silver medal at ACM ICPC World Finals 2017. I took advanced course on C++ during my study, check \url{https://github.com/safoex/nixcpp}.
\newline
\newline
Eurobot Finals are scheduled to the middle of May, so I'm available since June. I looked through the list of potential ideas for Google Summer of Code. I chose most interesting and relevant for me:
\begin{itemize}
	\item \textbf{Robot Work Cell Discovery} This project can improve many of my skills. I'd like to mention, that I have a small experience with UR3 robot from Robotics course at Skoltech. Project goal is actual for such collaborative robot. Also, expierence gained can be applied for other scene building tasks (e.g. for indoor mobile assistants). My future master thesis will be connected with mobile robots or with collaborative robots such as UR3. So, this is a great opportunity to work on this topic and finish it as part of my master thesis. 
	\item \textbf{Simulate Radar sensor} I've got an excellent background in physics (even won over MIT team in International Theoretical Physics Olympiad \url{http://thworldcup.com/contest2017} and attached NagaokaBags.pdf), so this project involves an efficient fusion of my skills! Also, it again gives me possible mobile robotics connected master thesis.
	\item \textbf{Create Default Grasping Library} Grasping is another \textit{hot} area of robotics. Experience in grasping is required for developing a good collaborative robot.  	
\end{itemize}
Personally I'm working on decision-making for Eurobot competiton. One of subtasks is a strategy optimization and execution. As for optimization part, that's individual for each project. For execution we decided to use behavior trees for their reactivity and human readability. There are several realisations of Behavior Tree concept in ROS:
\begin{itemize}
	\item \textbf{behavior\textunderscore tree} This one doesn't work on ROS Kinetic by default. I made few corrections in code (and in 2 depended packages, see again \url{https://github.com/SkoltechRobotics/ros-eurobot-2018} ROS-Behavior-tree, xdot and rqt\textunderscore dot packages). However, their approach didn't match our goals for Eurobot. It's not light-weight and huge BT initialisation and re-initialisation takes a long time. Majority of leaf nodes for mobile robots are 'moving' nodes and in my opinion this package isn't optimizied for fast-execution of similar (but not exactly the same) tasks.
	\item \textbf{decision\textunderscore making} pure C++ approach. 
	\item \textbf{pi\textunderscore trees}  I discovered it very recently and like their approach.
\end{itemize}
However, it still lacks some desired features.
I found all them miss certain desirable features:
\begin{itemize}
	\item Save and load BT-structure, construct tree from text/xml/json/etc description. This feature simplifies debug, work and online reconstruction of BT.
	\item Multi-agent BT-execution. Nowadays there is a many research and state-of-the-art work in robotics in the area of multi-agent cooperation.
	\item Optimized work in terms of service/topics/actionlib re-use. ROS is executed on Linux, but in many projects robot's actuators are not connected directly to Linux-carrying board. Usually, people use STM32/Arduino/etc board for low-level electronics. So, many leaf actions are just communication with this boards.
\end{itemize}

These features improves the value and actuality of BT framework. I've started to work on it and wrote a simple Python class for BT (\url{https://goo.gl/37b1hU}, Github). To finish this job in proper way will be very beneficial and joyful work for me. 
\newline
\newline
Of course, I'm open to any other suggested projects! I'm on the beginning of my engineering career and don't have any prejudicies against any robotics/software engineering areas.
\makeletterclosing

\end{document}


%% end of file `template.tex'.
