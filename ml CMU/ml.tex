%% start of file `template.tex'.
%% Copyright 2006-2013 Xavier Danaux (xdanaux@gmail.com).
%
% This work may be distributed and/or modified under the
% conditions of the LaTeX Project Public License version 1.3c,
% available at http://www.latex-project.org/lppl/.
%Version for spanish users, by dgarhdez

\documentclass[11pt,a4paper,roman]{moderncv}        % possible options include font size ('10pt', '11pt' and '12pt'), paper size ('a4paper', 'letterpaper', 'a5paper', 'legalpaper', 'executivepaper' and 'landscape') and font family ('sans' and 'roman')
\usepackage[english]{babel}


% moderncv themes
\moderncvstyle{classic}                            % style options are 'casual' (default), 'classic', 'oldstyle' and 'banking'
\moderncvcolor{green}                              % color options 'blue' (default), 'orange', 'green', 'red', 'purple', 'grey' and 'black'
%\renewcommand{\familydefault}{\sfdefault}         % to set the default font; use '\sfdefault' for the default sans serif font, '\rmdefault' for the default roman one, or any tex font name
%\nopagenumbers{}                                  % uncomment to suppress automatic page numbering for CVs longer than one page

% character encoding
\usepackage[utf8]{inputenc}                       % if you are not using xelatex ou lualatex, replace by the encoding you are using
%\usepackage{CJKutf8}                              % if you need to use CJK to typeset your resume in Chinese, Japanese or Korean

% adjust the page margins
\usepackage[scale=0.75]{geometry}
%\setlength{\hintscolumnwidth}{3cm}                % if you want to change the width of the column with the dates
%\setlength{\makecvtitlenamewidth}{10cm}           % for the 'classic' style, if you want to force the width allocated to your name and avoid line breaks. be careful though, the length is normally calculated to avoid any overlap with your personal info; use this at your own typographical risks...

% personal data
\name{Evgenii}{Safronov}
%\title{Resumé title}                               % optional, remove / comment the line if not wanted
%\address{Dirección}{CP, Ciudad}{País}% optional, remove / comment the line if not wanted; the "postcode city" and and "country" arguments can be omitted or provided empty
%\phone[mobile]{000-000-000-000}                   % optional, remove / comment the line if not wanted
%\phone[fixed]{+2~(345)~678~901}                    % optional, remove / comment the line if not wanted
%\phone[fax]{+3~(456)~789~012}                      % optional, remove / comment the line if not wanted
%\email{mailmailmail@gmail.com}                               % optional, remove / comment the line if not wanted
%\homepage{www.johndoe.com}                         % optional, remove / comment the line if not wanted
%\extrainfo{additional information}                 % optional, remove / comment the line if not wanted
%\photo[64pt][0.4pt]{picture}                       % optional, remove / comment the line if not wanted; '64pt' is the height the picture must be resized to, 0.4pt is the thickness of the frame around it (put it to 0pt for no frame) and 'picture' is the name of the picture file
%\quote{Some quote}                                 % optional, remove / comment the line if not wanted

% to show numerical labels in the bibliography (default is to show no labels); only useful if you make citations in your resume
%\makeatletter
%\renewcommand*{\bibliographyitemlabel}{\@biblabel{\arabic{enumiv}}}
%\makeatother
%\renewcommand*{\bibliographyitemlabel}{[\arabic{enumiv}]}% CONSIDER REPLACING THE ABOVE BY THIS

% bibliography with mutiple entries
%\usepackage{multibib}
%\newcites{book,misc}{{Books},{Others}}
%----------------------------------------------------------------------------------
%            content
%----------------------------------------------------------------------------------
\begin{document}
%-----       letter       ---------------------------------------------------------
% recipient data
\recipient{CMU}{Robotics Institute Summer Scholars Program}
\date{\today}
\opening{Dear CMU,}
\closing{Hope I delivered my sincere interest in CMU research and RISS.}
%\enclosure[Adjunto]{CV}          % use an optional argument to use a string other than "Enclosure", or redefine \enclname
\makelettertitle
First time I was impressed by CMU was on my last year of B.S., when I finally decided to go for MS in Robotics. In Russia education in robotics area is underdeveloped. So, it wasn't even hard to understand what I want from degree and find all courses/projects in one place. And when I accidently found CMU MSR program page it was \textit{exactly what I expected from proper MS Robotics program}! Nowadays I study in Skoltech and work  in Intelligent Space Robotics laboratory. My \textbf{BS with honors} (GPA 4.92 of 5) from Moscow Institute of Physics and Technology gave me strong background in math, physics and programming (however, I've studied much programming by myself since school years). I've even \textbf{won over MIT} in International Theoretical Physics olympiad as a member of Nagaoka Bags team! (see \url{http://thworldcup.com/contest2017} ) Accidentally, in last two years of my BS I worked on some robotics projects by myself and fell in love with robotics.
\newline
\newline
Second time was when I started to search for recent achievements in robotics. CMU figured out to have a lot of scientific influence on mobile robotics! My general dream is to create sort of really helpful indoor domestic or office/hotel/restaraunt robot. That's a sophisticated task and covers many areas - CV for objects/human detection, path planning (especially among crowds), indoor SLAM, pick-and-place actions, indoor barriers (e.g doorsteps, stairs) overcoming, multi-robot cooperation, human pose estimation, NLP, context-aware behavior and others. CMU has a plenty of relevant researches in mentioned areas and that was one of the main reasons for my RISS application. Of course, I can't become a professional in all required fields simultaneously. 
\newline
\newline 
Currently I'm \textbf{captain of Skoltech's Eurobot team}. Eurobot is probably the most popular and prestigious european robotics competition, it requires designing, building and programming new pair of mobile robots each year and trains various skills from project management to CV. This year main task are to build towers of colored cubes. Bonus color combination is unknown before start of the match -- and has to be determined by CV (checking some particular place of field). This fact, competition with pair of opponent robots and strict duration (100sec) of match introduce the need for flexible and reactive strategy, which is one of my tasks on Eurobot. We treated earning points during match as IP optimization problem. Result of such optimization is 2 behavior trees for each of 2 robots. I already wrote simple, lightweight behavior tree ROS implementation in Python for our CA. Robot strategy optimization was the project on Optimization Methods course for which I received A (excellent) grade. This approach looks good in situation with strict constraints about time and points, but I'd like to learn more about robot task scheduling, path planning and other aspects of decision making in uncertain environments (such as domestic/office/etc).
\newline
\newline
As I mentioned above, I looked through CMU RI Research website and found many relevant and inspiring areas of research. Here I want to mention labs \& groups, I'd like to work with. First is \textbf{Personal Robotics}. I wrote that I dream about \textit{proper} domestic/office robot and they are working exactly in the same direction! An opportunity to work in this lab will be truly a miracle for me. Next, \textbf{Manipulation Lab}, which can boost my knowledge and skills in grasping technics. I expect also a lot from Soft Robotics \& Bionics, soft manipulators in my opinion has a lot of potential for robust objects grasping. Than, Human Sensing and Computer Vision labs. High demand in their results raise my will to work in. A little bit aside of previous labs I'd like to mention  Intelligent Coordination and Logistics Laboratory. This lab doesn't fit my general dream about domestic/office robot, but last term I successfully passed Systems Engineering course and found all those concepts pretty interesting and valuable.
\newline
\newline
I'm really inspired by my current work on Eurobot. It improves my robotics, leadership and management skills. However, it's more or less training before master thesis. Such a great university as CMU and its RISS program shall be a good kickstart for my further research! \newline
\newline
%\vspace{0.5cm}


\makeletterclosing

\end{document}


%% end of file `template.tex'.
